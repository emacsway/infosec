\section{Атаки на протоколы}\label{section-protocols-attacks}
\selectlanguage{russian}

Защищённые свойства протоколов могут быть заявленными, когда о них заявляют сами авторы протокола (и, обычно, приводят различные аргументы в пользу выполнения данных функций), и подразумеваемыми, когда авторы некоторой системы рассчитывают на реализацию защищённых свойств некоторым протоколом.

Под \emph{атакой на защищённый протокол}\index{атака!на протокол} понимается попытка проведения анализа сообщений протокола и/или выполнения не\-пред\-усмотренных протоколом действий (уничтожение, дублирование или генерация и посылка сообщений) для нарушения заявленных или подразумеваемых свойств протокола.\footnote{Используется модифицированное определение из~\cite{Cheremushkin:2009}. Отличие в том, что Черёмушкин в своём определении не описывает, что такое <<нарушение работы протокола>> и оставляет двусмысленными случаи нарушения, например, свойств G9/PFS и G20/STP.}

Атака считается \emph{успешной}, если нарушено хотя бы одно из заявленных или подразумеваемых свойств протокола.

В случае успешной атаки на подразумеваемые свойства будем уточнять, что успешна \emph{атака на использование протокола} в некоторой системе. Это будет говорить, разумеется, не о недостатках самого протокола, но о неверном выборе протокола (или его настроек) авторами системы.

Существует большое количество типов атак на протоколы. У многих атак есть некоторые общие принципы, что позволяет выделить классы атак для упрощения анализа и разработки протоколов, устойчивых к целым классам атак.

\begin{itemize}
    \item[MITM] Атака <<человек посередине>>\index{атака!<<человек посередине>>}\\*
        \langen{man-in-the-middle attack}
    \item[{}] Класс атак, в котором злоумышленник ретранслирует и, при необходимости, изменяет все сообщения, проходящие между двумя и более участниками протокола, причём последние не знают о существовании злоумышленника, считая, что общаются непосредственно друг с другом. К данной атаке уязвимы все протоколы, которые не реализуют взаимную аутентификацию сторон (цель G1). Классическим примером атаки данного класса является атака на протокол Диффи~---~Хеллмана\index{протокол!Диффи~---~Хеллмана}, рассмотренном в разделе~\ref{section-protocols-diffie-hellman}.

    \item[Replay] Атака с повторной передачей\\*
        \langen{replay attack}
    \item[{}] Класс атак, в котором злоумышленник записывает все сообщения, проходящие в одном сеансе протокола, а далее повторяет их в новом, выдавая себя за одного из участников первого сеанса. Примерами протоколов, к которым применима данная атака, являются протоколы Ву~---~Лама\index{протокол!Ву~---~Лама} и бесключевой протокол Шамира\index{протокол!Шамира бесключевой} из раздела~\ref{section-protocols-shamir}.

    \item[TF] Атака подмены типа\\*
        \langen{type flaw attack}
    \item[{}] Класс атак, в котором злоумышленник используя переданное в легальном сеансе протокола сообщение конструирует новое, передавая его на другом проходе (раунде) протокола под видом сообщения другого типа (с другим предназначением). К таким атакам уязвимы, например, протоколы Wide-Mouth Frog\index{протокол!Wide-Mouth Frog} из раздела~\ref{section-protocols-wide-moth-frog}, Деннинг~---~Сакко\index{протокол!Деннинг~---~Сакко}, Отвей~---~Рииса\index{протокол!Отвей~---~Рииса}, а также некоторые варианты протокола Yahalom\index{протокол!Yahalom}.

    \item[PS] Атака с параллельными сеансами\\*
        \langen{parallel-session attack}
    \item[{}] Класс атак, в котором злоумышленник инициирует несколько одновременных сеансов протокола с целью использования сообщений из одного сеанса в другом. Примером протокола, уязвимого к данному классу атак, является симметричный вариант протокола Нидхема~---~Шрёдера\index{протокол!Нидхема~---~Шрёдера}, рассмотренном в разделе~\ref{section-protocols-needham-schroeder}.

    \item[STS] Атака с известным разовым ключом\index{атака!с известным разовым ключом}\\*
        \langen{short-term secret attack}
    \item[KN] Атака с известным сеансовым ключом\index{атака!с известным сеансовым ключом}\\*
        \langen{known-key attack}
    \item[{}] Классы атак, в которых злоумышленник получает доступ к временным секретам, используемых в протоколах (например, новым сеансовым ключам), после чего может обеспечить, например, аутентификацию или хотя бы установление сессии от имени одной из сторон протокола.

    \item[UKS] Атака с неизвестным сеансовым ключом\\*
        \langen{unknown key-share attack}
    \item[{}] Класс атак на протоколы с аутентификацией ключа, в которых злоумышленник получает возможность доказать одной из сторон владение ключом (с помощью, например, повтора сообщения из легального сеанса), хотя сам ключ злоумышленник не знает. К такому классу атак уязвим, например, симметричный протокол Нидхема-Шрёдера из раздела~\ref{section-protocols-needham-schroeder}.

\end{itemize}

Важно отметить, что если не сказано иное, то в рамках анализа криптографических протоколов (не конкретных систем) используется предположение о стойкости всех используемых криптографических примитивов. Например, предполагается, что пока идёт защищённый обмен информацией, использующий сеансовый ключ, выработанный в сеансе некоторого криптографического протокола, то злоумышленнику не хватит ресурсов и времени на то, чтобы получить данный сеансовый ключ через атаку на используемые шифры или криптографически-стойкие хеш-функции.

С другой стороны, следует предполагать, что сеансовые ключи, получаемые в рамках сеансов протоколов, через некоторое время (однако, много большее времени самого сеанса связи) будут получены злоумышленником (классы атак STS и KN). Много позднее, возможно, злоумышленник сможет получить доступ и к <<мастер>>-ключам длительного использования, так что протоколы с генерацией сеансовых ключей должны разрабатываться в том числе со свойством G9/PFS.
