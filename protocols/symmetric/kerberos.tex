\subsection{Протокол <<Kerberos>>}\index{протокол!Kerberos|(}\label{section-protocols-kerberos}
\selectlanguage{russian}

В данном разделе будет описан протокол аутентификации сторон с единственным доверенным центром. Сетевой протокол <<Ker\-be\-ros>> использует эти идеи при объединении нескольких доверенных центров в единую сеть для обеспечения надёжности и отказоустойчивости. Подробнее о сетевом протоколе <<Kerberos>> смотрите в разделе~\ref{section-kerberos}.

\begin{figure}
	\centering
	\begin{sequencediagram}
		\newinst{Alice}{Alice}
		\newinst[2.5]{Trent}{Trent}
		\newinst[2.5]{Bob}{Bob}

		\mess{Alice}{$ A, B $}{Trent}
		\mess{Trent}{$ E_A \left( T_T, L, K, B \right), E_B \left( T_T, L, K, A \right) $}{Alice}
		\mess{Alice}{$ E_B \left( T_T, L, K, A \right), E_K \left( A, T_A \right) $}{Bob}
		\mess{Bob}{$ E_K \left( T_T + 1 \right) $}{Alice}
	\end{sequencediagram}
	\caption{Протокол <<Kerberos>>\label{fig:key_distribution-kerberos}}
\end{figure}

Как и в протоколе Нидхема~---~Шрёдера, инициирующий абонент (Алиса) общается только с выделенным доверенным центром, получая от него два пакета с зашифрованным сессионным ключом -- один для себя, а второй -- для вызываемого абонента (Боба). Однако в отличие от Нидхема~---~Шрёдера\index{протокол!Нидхема~---~Шрёдера} в рассматриваемом протоколе зашифрованные пакеты содержат также метку времени $T_T$ и срок действия сессионного ключа $L$ (от \langen{lifetime} -- срок жизни). Что позволяет, во-первых, защититься от рассмотренной в предыдущем разделе атаки повтором. А, во-вторых, позволяет доверенному центру в некотором смысле управлять абонентами, заставляя их получать новые сессионные ключи по истечению заранее заданного времени $L$.

\begin{protocol}
	\item[(1)] $ Alice \to \{ A, B \} \to Trent $
	\item[(2)] $ Trent \to \{ E_A \left( T_T, L, K, B \right), E_B \left( T_T, L, K, A \right) \} \to Alice $
	\item[(3)] $ Alice \to \{ E_B \left( T_T, L, K, A \right), E_K \left( A, T_A \right) \} \to Bob $
	\item[(4)] $ Bob \to \{ E_K \left( T_T + 1 \right) \} \to Alice $
\end{protocol}

Обратите внимание, что на третьем проходе за счёт использования метки времени от доверенного центра $T_T$ вместо случайной метки от Боба $R_B$ позволяет сократить количество проходов на один по сравнению с протоколом Нидхема~---~Шрёдера\index{протокол!Нидхема~---~Шрёдера}. Также наличие метки времени делает ненужным и предварительную генерацию случайной метки Алисой и её передачу на первом шаге.

Метка времени $T_A$ в сообщении $E_K \left( A, T_A \right)$ позволяет Бобу убедиться, что Алиса владеет текущим сессионным ключом $K$. Если расшифрованная метка $T_A$ сильно отличается от текущего времени, значит либо этот пакет из другого сеанса протокола, либо не от Алисы вообще.

Пакеты $E_A \left( T_T, L, K, B \right)$ и $E_B \left( T_T, L, K, A \right)$ одинаковы по своему формату. В некотором смысле их можно назвать сертификатами сессионного ключа для Алисы и Боба. Причём все подобные пары пакетов можно сгенерировать заранее (например, в начале дня), выложить на общедоступный ресурс, предоставить в свободное использование и выключить доверенный центр (он своё дело уже сделал -- сгенерировав эти пакеты). И до момента времени $T_T + L$ этими <<сертификатами>> можно пользоваться. Но только если вы являетесь одной из допустимых пар абонентов. Конечно, эта идея непрактична -- ведь количество таких пар растёт как квадрат от числа абонентов. Однако интересен тот факт, что подобные пакеты можно сгенерировать заранее. Эта идея нам пригодится при рассмотрении инфраструктуры открытых ключей (\langen{public key infrastructure, PKI}).

\index{протокол!Kerberos|)}