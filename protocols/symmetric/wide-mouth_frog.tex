\subsection{Протокол Wide-Mouth Frog}\index{протокол!Wide-Mouth Frog|(}\label{section-protocols-wide-moth-frog}
Протокол Wide-Mouth Frog является, возможно, самым простым протоколом с доверенным центром. Его автором считается Майкл Бэрроуз (1989 год, \langen{Michael Burrows}, \cite{Burrows:Abadi:Needham:1990}, \ref{fig:key_distribution-wide-mouth_frog}). Протокол состоит из следующих проходов.

\begin{figure}
	\centering
	\begin{sequencediagram}
	    \newinst{Alice}{Alice}
	    \newinst[3]{Trent}{Trent}
	    \newinst[3]{Bob}{Bob}

	    \mess{Alice}{$ A, E_A \left( T_A, B, K \right)$}{Trent}
	    \mess{Trent}{$E_B \left( T_T, A, K \right)$}{Bob}
	\end{sequencediagram}
    \caption{Диаграмма последовательности взаимодействия абонентов и доверенного центра в протоколе Wide-Mouth Frog\label{fig:key_distribution-wide-mouth_frog}}
\end{figure}

\begin{protocol}
	\item[(1)] Алиса генерирует новый сеансовый ключ $K$
    \item[{}] $Alice \to \{ A, E_A \left( T_A, B, K \right) \} \to Trent$
	\item[(2)] $Trent \to \{ E_B \left( T_T, A, K \right) \} \to Bob$
\end{protocol}

По окончании протокола у Алисы и Боба есть общий сеансовый ключ $K$.

У данного протокола множество недостатков.

\begin{itemize}
	\item Генератором ключа является инициирующий абонент. Если предположить, что Боб -- это сервер, предоставляющий некоторый сервис, а Алиса -- это тонкий клиент, запрашивающий данный сервис, получается, что задача генерации надёжного сессионного ключа взваливается на плечи абонента с наименьшими мощностями.
	\item В протоколе общение с вызываемым абонентом происходит через доверенный центр. Как следствие, второй абонент может стать мишенью для DDOS-атаки с отражением (\langen{distributed denial-of-service attack}), когда злоумышленник будет посылать пакеты на доверенный центр, а тот формировать новые пакеты и посылать их абоненту. Если абонент подключён к нескольким сетям (с несколькими доверенными центрами), это позволит злоумышленнику вывести абонента из строя. Хотя защититься от подобной атаки достаточно просто, настроив соответствующим образом доверенный центр.
\end{itemize}

Однако самой серьёзной проблемой протокола является возможность применения следующих атак.

В 1995 году Рос Андерсон и Роджер Нидхем (\langen{Ross Anderson, Roger Needham}, \cite{Anderson:Needham:1995}, рис~\ref{fig:key_distribution-wide-mouth_frog-Anderson-Needham-attack}) предложили вариант атаки на протокол, при котором злоумышленник (Ева) может бесконечно продлевать срок действия конкретного сеансового ключа. Идея атаки в том, что после окончания протокола злоумышленник будет посылать доверенному центру назад его же пакеты (перехваченные ранее), дополняя их идентификаторами якобы инициирующего абонента.

\begin{figure}
	\centering
	\begin{sequencediagram}
		\newinst{Alice}{Alice}
		\newinst[1]{Trent}{Trent}
		\newinst[1]{Eva}{Eva}
		\newinst[1]{Bob}{Bob}
		
		\mess{Alice}{$  A, E_A \left( T_A, B, K \right) $}{Trent}
		\mess{Trent}{$ E_B E_B \left( T_T, A, K \right) $}{Bob}
		\mess{Eva}{$ B, E_B \left( T_T, A, K \right) $}{Trent}
		\mess{Trent}{$ E_A \left( T'_T, B, K \right) $}{Alice}
		\mess{Eva}{$ A, E_A \left( T'_T, B, K \right) $}{Trent}
        \mess{Trent}{$ E_B \left( T''_T, A, K \right) $}{Bob}
	\end{sequencediagram}
	\caption{Диаграмма последовательности атаки Роса Андерсона и Роджера Нидхема на протокол <<Wide-Mouth Frog>>\label{fig:key_distribution-wide-mouth_frog-Anderson-Needham-attack}}
\end{figure}

\begin{protocol}
	\item[(1)] $Alice \to \{ A, E_A \left( T_A, B, K \right) \} \to Trent$
	\item[(2)] $Trent \to \{ E_B \left( T_T, A, K \right) \} \to Bob$
	\item[(3)] $Eva \to \{ B, E_B \left( T_T, A, K \right) \} \to Trent$
	\item[(4)] $Trent \to \{ E_A \left( T'_T, B, K \right) \} \to Alice$
	\item[(5)] $Eva \to \{ A, E_A \left( T'_T, B, K \right) \} \to Trent$
	\item[(6)] $Trent \to \{ E_B \left( T''_T, A, K \right) \} \to Bob$
	\item[{}] Повторять проходы 3 и 5, пока не пройдёт время, нужное для получения $K$.
\end{protocol}

С точки зрения доверенного центра, шаги 1, 3 и 5 являются корректными пакетами, инициирующими общение абонентов между собой. Метки времени в них корректны (если Ева будет успевать вовремя эти пакеты посылать). С точки зрения легальных абонентов каждый из пакетов является приглашением другого абонента начать общение. В результате произойдёт две вещи:

\begin{itemize}
	\item Каждый из абонентов будет уверен, что закончился протокол создания нового сеансового ключа, что второй абонент успешно аутентифицировал себя перед доверенным центром. И что для установления следующего сеанса связи будет использоваться новый (на самом деле -- старый) ключ $K$.
	\item После того, как пройдёт время, нужное злоумышленнику Еве для взлома сеансового ключа $K$, Ева сможет и читать всю переписку, проходящую между абонентами, и успешно выдавать себя за обоих из абонентов.
\end{itemize}

Вторая атака 1997 года Гэвина Лоу (\langen{Gavin Lowe}, \cite{Lowe:1997}, рис.~\ref{fig:key_distribution-wide-mouth_frog-lowe-attack}) проще в реализации. В результате этой атаки Боб уверен, что Алиса аутентифицировала себя перед доверенным центром и хочет начать второй сеанс общения. Что, конечно, не является правдой, так как второй сеанс инициирован злоумышленником.

\begin{figure}
	\centering
	\begin{sequencediagram}
		\newinst{Alice}{Alice}
		\newinst[1]{Trent}{Trent}
		\newinst[1]{Eva}{Eva}
		\newinst[1]{Bob}{Bob}

		\mess{Alice}{$ A, E_A \left( T_A, B, K \right) $}{Trent}
		\mess{Trent}{$ E_B \left( T_T, A, K \right) $}{Bob}
		\mess{Eva}{$ E_B \left( T_T, A, K \right) $}{Bob}
	\end{sequencediagram}
	\caption{Диаграмма последовательности атаки Гэвина Лоу на протокол <<Wide-Mouth Frog>>\label{fig:key_distribution-wide-mouth_frog-lowe-attack}}
\end{figure}

\begin{protocol}
	\item[(1)] $Alice \to \{ A, E_A \left( T_A, B, K \right) \} \to Trent$
	\item[(2)] $Trent \to \{ E_B \left( T_T, A, K \right) \} \to Bob$
	\item[(3)] $Eva \to \{ E_B \left( T_T, A, K \right) \} \to Bob$
\end{protocol}

В той же работе Лоу предложил модификацию протокола, вводящую явную взаимную аутентификацию абонентов с помощью случайной метки $R_B$ и проверки, что Алиса может расшифровать пакет с меткой, зашифрованной общим сеансовым ключом абонентов $K$. Однако данная модификация приводит к тому, что протокол теряет своё самое главное преимущество перед другими протоколами -- простоту.

\begin{protocol}
	\item[(1)] $Alice \to \{ A, E_A \left( T_A, B, K \right) \} \to Trent$
	\item[(2)] $Trent \to \{ E_B \left( T_T, A, K \right) \} \to Bob$
	\item[(3)] $Bob \to \{ E_K \left( R_B \right) \} \to Alice$
	\item[(4)] $Alice \to \{ E_K \left( R_B + 1 \right) \} \to Bob$
\end{protocol}

\index{протокол!Wide-Mouth Frog|)}
