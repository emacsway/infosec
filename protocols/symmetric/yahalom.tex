\subsection{Протокол Yahalom}\index{протокол!Yahalom|(}\label{section-protocols-yahalom}

Протокол Yahalom можно рассматривать как улучшенную версию протокола Wide-Mouth Frog\index{протокол!Wide-Mouth Frog} из раздела~\ref{section-protocols-wide-moth-frog}. Данный протокол <<перекладывает>> генерацию нового сессионного ключа на сторону доверенного центра, а также использует случайные числа для защиты от атак повтором.

\begin{figure}[thb]
	\centering
	\begin{sequencediagram}
		\newinst{Alice}{Alice}
		\newinst[3]{Trent}{Trent}
		\newinst[3]{Bob}{Bob}

		\mess{Alice}{$A$, $R_A$}{Bob}
		\mess{Bob}{$ B, E_B( A, R_A, R_B ) $}{Trent}
		\mess{Trent}{$ E_A( B, K, R_A, R_B )$, $E_B(A, K) $}{Alice}
		\mess{Alice}{$E_B( A, K )$, $E_K( R_B )$}{Bob}
	\end{sequencediagram}
	\caption{Диаграмма последовательности взаимодействия абонентов и доверенного центра в протоколе Yahalom\label{fig:key_distribution-yahalom}}
\end{figure}

\begin{protocol}
	\item[(1)] $Alice \to \{ A, R_A \} \to Bob$
	\item[(2)] $Bob \to \{ B, E_B( A, R_A, R_B ) \} \to Trent$
	\item[(3)] Трент генерирует новый сессионный ключ $K$
	\item[{}] $Trent \to \{ E_A( B, K, R_A, R_B ), E_B(A, K) \} \to Alice$
	\item[(4)] $Alice \to \{ E_B( A, K ), E_K( R_B ) \} \to Bob$
\end{protocol}

После того, как Боб провалидирует число $R_B$, присланное Алисой, стороны смогут использовать новый сессионный ключ $K$. Протокол, кроме генерации ключа, обеспечивает взаимную аутентификацию сторон:

\begin{itemize}
    \item Аутентификация Алисы перед Бобом происходит на 4-м проходе, когда Боб может провалидировать возможность Алисы зашифровать известное только ей (и Тренту) случайное число $R_B$ на ключе $K$.
    \item Аутентификация Боба перед Алисой происходит на 3-м проходе, когда Трент демонстрирует Алисе, что он получил случайное число $R_A$ именно от Боба.
\end{itemize}

Нужно отметить (\cite{Zhou:Yu:Pan:Wang:2016}), что в рамках протокола Боб никак не продемонстрировал, что он успешно получил новый сессионный ключ $K$ и может им оперировать (не выполнена цель G8). Сообщение от Алисы на 4-м проходе могло быть перехвачено или модифицировано злоумышленником. Но никакого ответа Алиса от Боба уже не ожидает и уверена, что протокол завершился успешно.

Также на 3-м проходе Трент не включает случайное число $R_B$ в сообщение $E_B(A, K)$, что позволяет Алисе, действуя не из лучших побуждений, заставить Боба принять старый сессионный ключ (рис.~\ref{fig:key_distribution-yahalom-attack}).

\begin{figure}[thb]
	\centering
	\begin{sequencediagram}
		\newinst{Alice}{Alice}
		\newinst[3]{Trent}{Trent}
		\newinst[3]{Bob}{Bob}

		\mess{Alice}{$A$, $R_{A2}$}{Bob}
		\mess{Bob}{$ B, E_B( A, R_{A2}, R_{B2} ) $}{Trent}
		\mess{Trent}{$ E_A( B, K_2, R_{A2}, R_{B2} )$, $E_B(A, K_2) $}{Alice}
		\mess{Alice}{$E_B( A, K_1 )$, $E_{K_1}( R_{B2} )$}{Bob}
	\end{sequencediagram}
	\caption{Диаграмма последовательности атаки на протокол Yahalom для использованием старого сессионного ключа\label{fig:key_distribution-yahalom-attack}}
\end{figure}

\begin{protocol}
	\item[(1)] $Alice \to \{ A, R_A \} \to Bob$
	\item[(2)] $Bob \to \{ B, E_B( A, R_A, R_B ) \} \to Trent$
	\item[(3)] Трент генерирует новый сессионный ключ $K_2$
	\item[{}] $Trent \to \{ E_A( B, K, R_A, R_B ), E_B(A, K_2) \} \to Alice$
	\item[(4)] Алиса использует старый сессионный ключ $K_1$ и сообщение $E_B( A, K_1 )$ из старого сеанса протокола
	\item[{}] $Alice \to \{ E_B( A, K_1 ), E_{K_2}( R_B ) \} \to Bob$
\end{protocol}

Протокол Yahalom послужил основной большому количеству научных работ, связанных с автоматизированным анализом стойкости криптографических протоколов и имел несколько <<улучшенных>> вариантов. Однако о широком использовании данного протокола в реальных информационных системах неизвестно.

\index{протокол!Yahalom|)}
