\section{Курс <<Криптографические протоколы>>}
\selectlanguage{russian}

Список вопросов по курсу <<Криптографические протоколы>> кафедры защиты информации МФТИ для 5-го курса.

\begin{enumerate}
    \item Криптографические протоколы. Определения, способы записи, классификация.
    \item Свойства безопасности криптографических протоколов в терминах проекта AVISPA. С примерами реализации свойств, относящихся к распределению ключей.
    \item Атаки на криптографические протоколы. С примерами.

    \item Протокол Wide-Mouth Frog. Описание, плюсы и минусы, возможные атаки, известные модификации.
    \item Протокол Yahalom. Описание, плюсы и минусы, возможные атаки, известные модификации.
    \item Симметричный вариант протокола Нидхема~---~Шрёдера. Описание, плюсы и минусы, возможные атаки, известные модификации.
    \item Симметричный вариант протокола Kerberos (математический). Описание, плюсы и минусы, возможные атаки, известные модификации.
    \item Трёхпроходные протоколы. Бесключевой протокол Шамира. Описание, плюсы и минусы, возможные атаки.
    \item Протокол Диффи~---~Хеллмана. Описание, плюсы и минусы, возможные атаки, известные модификации.
    \item Протокол Эль-Гамаля. Описание, плюсы и минусы, возможные атаки, известные модификации.
    \item Протокол MTI/A(0). Описание, плюсы и минусы, возможные атаки, известные модификации.
    \item Протокол Station-to-Station. Описание, плюсы и минусы, возможные атаки, известные модификации.
    \item Схема Жиро. Описание, плюсы и минусы, возможные атаки, известные модификации.
    \item Схема Блома. Описание, плюсы и минусы, возможные атаки, известные модификации.
    \item Протокол Деннинг~---~Сакко. Описание, плюсы и минусы, возможные атаки, известные модификации.
    \item Протокол DASS. Описание, плюсы и минусы, возможные атаки, известные модификации.
    \item Протокол Ву~---~Лама. Описание, плюсы и минусы, возможные атаки, известные модификации.
    \item Протокол BB84. Описание, плюсы и минусы, возможные атаки, известные модификации.
    \item Протокол B92 / BB92. Описание, плюсы и минусы, возможные атаки, известные модификации.

    \item База данных на основе Echo-сети. Blockchain. Доказательство работы (proof-of-work, proof-of-share). BitCoin.
    \item Аутентификация в WEB. Basic, Digest, form- и cookie-ау\-тен\-ти\-фи\-ка\-ция, местоположение аутентификационных данных в HTTP-сообщении (path, headers, body). Использование HTTPS для аутентификации клиента.
    \item Аутентификация в WEB. Протокол OAuth 2.0. Использование JWT.

\end{enumerate}
